\documentclass[a4paper,10pt]{article}
\usepackage{amssymb,amsmath,amsthm,eurosym,booktabs}
\usepackage[top=3cm, bottom=4cm, left=2.5cm, right=2.5cm]{geometry}
%\usepackage{graphicx}
%\usepackage{pstricks}
\usepackage{natbib}
%\usepackage{egameps}
\usepackage[utf8x]{inputenc}
\usepackage{comment}

\newtheorem{hyp}{Hypothesis}
\newtheorem{fig}{Figure}
\newtheorem{tab}{Table}
\newtheorem{hypo}{Hypothesis}
\newtheorem{subhypo}{Hypothesis}[hypo]

\linespread{1.5}

%\renewcommand{\baselinestretch}{1.5}

%\bibliographystyle{harvard}

\title{\begin{LARGE}Literature overview\end{LARGE}}

\author{Tobias Regner \footnote{Delivery Hero
}~\\
\\
\\
   } % \vspace{1cm}

\date{ \today}

%%%%%%%%%% HEADERS %%%%%%%%%%
%\lhead{} \chead{} \rhead{} \lfoot{} \cfoot{\thepage}
%\rfoot{\begin{rotate}{15}
%  \texttt{\begin{Huge} draft :-))!\end{Huge}}%remove when it is OK
%  \end{rotate}}
%%%%%%%%%%%%%%%%%%%%%%%%%%%%%%

\begin{document}



\maketitle

\vspace{2.5cm}

\begin{center}
Summary:
\end{center}


Overview of the existing literature on experimentation, with a particular focus on variance reduction techniques.

\newpage

\section{Publications in refereed journals}

\begin{itemize}

\item Do Consumers Pay Voluntarily? The Case of Online Music.

\begin{itemize}
\item Journal of Economic Behavior \& Organization, 2009 (71), 395-406; Google Scholar cites: 204
\item Regner, Tobias and Barria, Javier A.

\item Abstract: \tiny The paper analyses the payment behaviour of customers of the online music label Magnatune. Customers may pay what they want for albums, as long as the payment is within a given price range (\$5-\$18). Magnatune's comprehensive pre-purchase access facilitates music discovery and allows an informed buying decision setting it apart from conventional online music stores. On average customers pay \$8.20, far more than the minimum of \$5 and even higher than the recommended price of \$8. We analyse the relationship between artists/labels and customers in online music. We consider social preferences, in particular concerns for reciprocity. The resulting sequential reciprocity equilibrium corresponds to the observed pattern of behaviour. We conclude that Magnatune's open contracts design can encourage people to make voluntary payments and may be a viable business option.

\end{itemize}
\end{itemize}


%\begin{comment}

	
\section{Chapters in books}

\begin{itemize}

\item Feldexperimente in der \"{O}konomik

\begin{itemize}
\item Jahrbuch 2010 der Max-Planck-Gesellschaft
\item Koppel, Hannes und Regner, Tobias
\item Abstract: \tiny In den letzten zwei Jahrzehnten haben sich Laborexperimente als Forschungsmethode in der \"{O}konomik etabliert. Seit kurzem werden auch verst\"{a}rkt Feldexperimente verwendet. Beide Methoden haben ihre Vor- und Nachteile, wobei wir deren Verwendung als komplement\"{a}r ansehen. Abschliessend stellen wir drei Studien \"{u}ber Zahlungs- und Spendenmechanismen vor, die als Beispiele f\"{u}r die Anwendung von Feldexperimenten stehen.
\end{itemize}

\item Wie die experimentelle Wirtschaftsforschung die Grundlagen des Vertrauens auslotet

\begin{itemize}
\item Jahrbuch 2013 der Max-Planck-Gesellschaft
\item Regner, Tobias und Winter, Fabian
\item Abstract: \tiny Ohne Vertrauen l\"{a}uft in der Wirtschaft wenig. Autos bleiben auf dem Hof des H\"{a}ndlers stehen, oder die Auswahl geeigneter Stellenbewerber w\"{u}rde sich sehr viel schwieriger gestalten. \"{O}konomen argumentieren dann gemeinhin,  dass zwar vieles in Vertr\"{a}ge festgeschrieben werden kann, aber doch eben nicht alle F\"{a}lle abgedeckt werden. Genau diese Eventualit\"{a}ten aber erfordern beiderseitiges Vertrauen. Die Experimental\"{o}konomen des Max-Planck-Institut f\"{u}r \"{O}konomik haben in einer Reihe von Experimenten untersucht, wann und warum wir in solchen Situationen vertrauen bzw. entgegengebrachtes Vertrauen erwidern.
\end{itemize}

\end{itemize}

%\end{comment}


\section{Work in Progress}
%\section{Publications as working papers}

\begin{itemize}

\begin{comment}

\item Digital Technology and the Allocation of Ownership in the Music Industry

\begin{itemize}
\item Halonen-Akatwijuka, Maija and Regner, Tobias
\item Google Scholar cites: 17
\item Abstract: \tiny We apply the property rights theory of Grossman-Hart-Moore in the music industry and study the optimal allocation of copyright between the artists who create music and the labels who promote and distribute it. Digital technology opens up a role for new intermediaries. We find that entry of online distribution businesses occurs only if they are sufficiently more productive in distribution than the incumbent label. Furthermore, entry leads to a change in bargaining positions and it can become optimal for the copyright to be shifted from the label to the artist.
\end{itemize}



\item Motivational Cherry Picking: Social Context and Moral Decisions

\begin{itemize}
\item Regner, Tobias and Riener, Gerhard
\item Google Scholar cites: 6
%\item Submission status: under preparation
\item Abstract: \tiny Moral decisions are often not taken in isolation but within a social context. This paper tests the effect of social context on moral decisions in a simple three person trust game. The trustor has the possibility to either trust both trustees or none, and each of the trustees make an independent decision to return trust. We experimentally vary the institutional environment, (a) letting both trustees decide in isolation, (b) providing a follower trustee with information on the behavior of a leader trustee and (c) a setting in which the follower trustee can condition her choice on the decision of the leader trustee. Follower trustees in the social context environments behave significantly more selfish - independent of the leader trustee's decision - than trustees who either decide in isolation or make a conditional decision. They appear to cherry pick the motivation that materially serves them best: When the leader trustee plays selfish, they tend to conform; when the leader makes a pro-social choice, followers seem to perceive the responsibility towards the trustor as already fulfilled by the leader.
\end{itemize}


\item Testing belief-dependent models

\begin{itemize}
\item Regner, Tobias and Harth, Nicole S.
\item Google Scholar cites: 6
%\item Submission status: under review (Journal of Economic Behavior \& Organization)
\item Abstract: \tiny We analyse two types of belief-dependant models of social preferences: guilt aversion and reciprocity. Our experimental data confirm their predictions. Both second-order beliefs and participants' dispositions (to guilt/reciprocity) are relevant for the decisions taken. Second-order beliefs appear to have an inverse U-shaped effect on the back transfer in a trust game. We also use a heterogeneous reference point based on trustees' first-order beliefs in order to assess perceived kindness of the trustor, and find that the difference between actual transfer and beliefs matters. The effect of disappointment (lower transfer than expected) is more pronounced than the one of a positive surprise.

\end{itemize}

\end{comment}




\item Actions and the self: I give, therefore I am?

\begin{itemize}
\item Regner, Tobias and Matthey, Astrid 
%\item Google Scholar cites: 
\item Submission status: revise \& resubmit (Journal of Economic Behavior \& Organization)
\item Abstract: \tiny Self-signaling models predict less selfish behavior in a probabilistic giving setting as individuals are expected to invest in a pro-social identity. However, there is also substantial evidence that people tend to exploit situational excuses for selfish choices (for instance, uncertainty) and behave more selfishly. We contrast these two motivations experimentally in order to test which one is more prevalent in a reciprocal giving setting. Trustees' back transfer choices are elicited for five different transfer levels of the trustor. Moreover, we ask trustees to provide their back transfer schedule for different scenarios that vary the implementation probability of the back transfer. This design allows us to identify subjects who reciprocate and analyze how these reciprocators respond when self-image relevant factors are varied. Our results indicate that self-deception is prevalent when subjects make the back transfer choice. Twice as many subjects seem to exploit situational excuses than subjects who appear to invest in a pro-social identity. 

\end{itemize}
\item The social pay gap across occupations: Experimental evidence

\begin{itemize}
\item Bublitz, Elisabeth and Regner, Tobias
%\item Google Scholar cites: 
\item Submission status: revise \& resubmit (Journal of Behavioral and Experimental Economics)
\item Abstract: \tiny Receiving equal wages for work of equal value is a legal right in many countries. However, it remains unknown to what degree the neglect of this principle yields differences in pay between social and other occupations. The results of a task-based analysis with survey data confirm a notable wage penalty of 0.5 standard deviations for social occupations (e.g., health care, education). Based on these results, we design a laboratory experiment that mimics actual income distributions (Germany, USA), incorporates social occupations in the lab society, and allows for (voluntary) redistribution among subjects. The results show that, regardless of (non-)random assignment to social jobs and the level of income inequality, individuals in social jobs are only partly compensated for their social effort. A downward spiral, induced by emotional reactions, results as social effort and donations converge to a `low' equilibrium. This suggests that a market approach fails to eliminate the social pay gap.


\end{itemize}



\end{itemize}



\newpage
% \bibliography

\addcontentsline{toc}{section}{Bibliography}
\linespread{1}
\bibliographystyle{aea}
\bibliography{lit_exp}


\end{document} 
